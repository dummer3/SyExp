% Created 2022-05-13 ven. 22:31
% Intended LaTeX compiler: pdflatex
\documentclass[11pt]{article}
\usepackage[utf8]{inputenc}
\usepackage[T1]{fontenc}
\usepackage{graphicx}
\usepackage{grffile}
\usepackage{longtable}
\usepackage{wrapfig}
\usepackage{rotating}
\usepackage[normalem]{ulem}
\usepackage{amsmath}
\usepackage{textcomp}
\usepackage{amssymb}
\usepackage{capt-of}
\usepackage{hyperref}
\usepackage[usenames,dvipsnames]{color}
\usepackage{listings}
\usepackage[a4paper,left=2cm,right=2cm,top=2cm,bottom=2cm]{geometry}
\author{cliquot}
\date{\today}
\title{}
\hypersetup{
 pdfauthor={cliquot},
 pdftitle={},
 pdfkeywords={},
 pdfsubject={},
 pdfcreator={Emacs 26.3 (Org mode 9.4.6)}, 
 pdflang={English}}
\begin{document}

\tableofcontents

\section{Exercice n°1}
\label{sec:orgd65a0ff}
\lstset{morekeywords={main, printf, include, week_t, action_t, Pile,
File, Arbre},backgroundcolor=\color[rgb]{0.96,0.95,0.98},keywordstyle=\color[rgb]{0.627,0.126,0.941},commentstyle=\color[rgb]{0.233,0.8,0.233},stringstyle=\color[rgb]{1,0,0},keepspaces=true,deletekeywords={ps,scan},basicstyle=\ttfamily,numbers=left,breaklines=true,frame=lines,tabsize=4,language=sh,label= ,caption= ,captionpos=b}
\begin{lstlisting}
echo $...
\end{lstlisting}
\begin{itemize}
\item \$\$ : Process ID (PID) du script actuel
\item \$BASHPID : PID du bash actuel
\item \$PPID : PID du processus parent
\item ! : (process ID) du dernier processus en background
\item OLDPWD : chemin du dossier précédent
\item PWD : chemin du dossier actuel
\item HOME : chemin vers notre dossier home
\item PATH : ensemble des chemins que le bash va regarder automatiquement
\item ? : status de retour de la commande, fonction, ou script lui même
\item PIPESTATUS : tableau conteanant les status de retour of the last executed foreground pipe.
\item RANDOM : nbr entier positif aléatoire
\item SRANDOM : pareil mais ne peut pas être reseemed
\item IFS : champ séparateur
\item PS1 : main prompt, du command-line.
\item SHELL : le chemin du bash (/bin/bash)
\item COLUMNS : Colonne du terminal
\item LINES : lignes du terminal
\end{itemize}


\lstset{morekeywords={main, printf, include, week_t, action_t, Pile,
File, Arbre},backgroundcolor=\color[rgb]{0.96,0.95,0.98},keywordstyle=\color[rgb]{0.627,0.126,0.941},commentstyle=\color[rgb]{0.233,0.8,0.233},stringstyle=\color[rgb]{1,0,0},keepspaces=true,deletekeywords={ps,scan},basicstyle=\ttfamily,numbers=left,breaklines=true,frame=lines,tabsize=4,language=sh,label= ,caption= ,captionpos=b}
\begin{lstlisting}
echo "PID:" $$ "BASHPID:" $BASHPID "PPID:" $PPID ==> 5916 5916 3302
\end{lstlisting}

\lstset{morekeywords={main, printf, include, week_t, action_t, Pile,
File, Arbre},backgroundcolor=\color[rgb]{0.96,0.95,0.98},keywordstyle=\color[rgb]{0.627,0.126,0.941},commentstyle=\color[rgb]{0.233,0.8,0.233},stringstyle=\color[rgb]{1,0,0},keepspaces=true,deletekeywords={ps,scan},basicstyle=\ttfamily,numbers=left,breaklines=true,frame=lines,tabsize=4,language=sh,label= ,caption= ,captionpos=b}
\begin{lstlisting}
(echo "PID:" $$ "BASHPID" $BASHPID "PPID" $PPID)
PID: 5916 BASHPID 7034 PPID 3032

\end{lstlisting}

\lstset{morekeywords={main, printf, include, week_t, action_t, Pile,
File, Arbre},backgroundcolor=\color[rgb]{0.96,0.95,0.98},keywordstyle=\color[rgb]{0.627,0.126,0.941},commentstyle=\color[rgb]{0.233,0.8,0.233},stringstyle=\color[rgb]{1,0,0},keepspaces=true,deletekeywords={ps,scan},basicstyle=\ttfamily,numbers=left,breaklines=true,frame=lines,tabsize=4,language=sh,label= ,caption= ,captionpos=b}
\begin{lstlisting}
{______ echo "PID:" $$ "BASHPID" $BASHPID "PPID" $PPID; _____}
PID: 5916 BASHPID 5916 PPID 3032
\end{lstlisting}

Pour voir le manuel de read est type, il faut regarder le man de builtins

\subsection{voir \textasciitilde{}/scripts/[readin/fctToFile/fctInsFile]}
\label{sec:org9521ea8}

\lstset{morekeywords={main, printf, include, week_t, action_t, Pile,
File, Arbre},backgroundcolor=\color[rgb]{0.96,0.95,0.98},keywordstyle=\color[rgb]{0.627,0.126,0.941},commentstyle=\color[rgb]{0.233,0.8,0.233},stringstyle=\color[rgb]{1,0,0},keepspaces=true,deletekeywords={ps,scan},basicstyle=\ttfamily,numbers=left,breaklines=true,frame=lines,tabsize=4,language=sh,label= ,caption= ,captionpos=b}
\begin{lstlisting}
 function fct() \n
{... ; ... ; done}

\end{lstlisting}


Puis type fct

Enfin pour tout mettre dans un fichier type fct | tail -n +2 > file

>;< \ldots{} = fichier

\begin{center}
\begin{tabular}{l}
= commande\\
\end{tabular}
\end{center}

\section{Exercice n°2}
\label{sec:orgff15c57}

\subsection{Question 1}
\label{sec:org506194c}

\lstset{morekeywords={main, printf, include, week_t, action_t, Pile,
File, Arbre},backgroundcolor=\color[rgb]{0.96,0.95,0.98},keywordstyle=\color[rgb]{0.627,0.126,0.941},commentstyle=\color[rgb]{0.233,0.8,0.233},stringstyle=\color[rgb]{1,0,0},keepspaces=true,deletekeywords={ps,scan},basicstyle=\ttfamily,numbers=left,breaklines=true,frame=lines,tabsize=4,language=sh,label= ,caption= ,captionpos=b}
\begin{lstlisting}
printf "Je m'appelle %s.\n" $USER
\end{lstlisting}

Affiche notre nom d'utilisateur

\subsection{Question 2}
\label{sec:orgb201461}

\lstset{morekeywords={main, printf, include, week_t, action_t, Pile,
File, Arbre},backgroundcolor=\color[rgb]{0.96,0.95,0.98},keywordstyle=\color[rgb]{0.627,0.126,0.941},commentstyle=\color[rgb]{0.233,0.8,0.233},stringstyle=\color[rgb]{1,0,0},keepspaces=true,deletekeywords={ps,scan},basicstyle=\ttfamily,numbers=left,breaklines=true,frame=lines,tabsize=4,language=sh,label= ,caption= ,captionpos=b}
\begin{lstlisting}
  strace -- printf "Je m'appelle %s.\n" $USER


execve("/usr/bin/printf", ["printf", "Je m'appelle %s.\\n", "cliquot"], 0x7ffdc488dc68 /* 53 vars */) = 0
brk(NULL)                               = 0x55b5742ae000
arch_prctl(0x3001 /* ARCH_??? */, 0x7ffd11cf63c0) = -1 EINVAL (Argument invalide)
access("/etc/ld.so.preload", R_OK)      = -1 ENOENT (Aucun fichier ou dossier de ce type)
openat(AT_FDCWD, "/etc/ld.so.cache", O_RDONLY|O_CLOEXEC) = 5
fstat(5, {st_mode=S_IFREG|0644, st_size=106506, ...}) = 0
mmap(NULL, 106506, PROT_READ, MAP_PRIVATE, 5, 0) = 0x7f58afc1f000
close(5)                                = 0
openat(AT_FDCWD, "/lib/x86_64-linux-gnu/libc.so.6", O_RDONLY|O_CLOEXEC) = 5
read(5, "\177ELF\2\1\1\3\0\0\0\0\0\0\0\0\3\0>\0\1\0\0\0\360A\2\0\0\0\0\0"..., 832) = 832
pread64(5, "\6\0\0\0\4\0\0\0@\0\0\0\0\0\0\0@\0\0\0\0\0\0\0@\0\0\0\0\0\0\0"..., 784, 64) = 784
pread64(5, "\4\0\0\0\20\0\0\0\5\0\0\0GNU\0\2\0\0\300\4\0\0\0\3\0\0\0\0\0\0\0", 32, 848) = 32
pread64(5, "\4\0\0\0\24\0\0\0\3\0\0\0GNU\0\237\333t\347\262\27\320l\223\27*\202C\370T\177"..., 68, 880) = 68
fstat(5, {st_mode=S_IFREG|0755, st_size=2029560, ...}) = 0
mmap(NULL, 8192, PROT_READ|PROT_WRITE, MAP_PRIVATE|MAP_ANONYMOUS, -1, 0) = 0x7f58afc1d000
pread64(5, "\6\0\0\0\4\0\0\0@\0\0\0\0\0\0\0@\0\0\0\0\0\0\0@\0\0\0\0\0\0\0"..., 784, 64) = 784
pread64(5, "\4\0\0\0\20\0\0\0\5\0\0\0GNU\0\2\0\0\300\4\0\0\0\3\0\0\0\0\0\0\0", 32, 848) = 32
pread64(5, "\4\0\0\0\24\0\0\0\3\0\0\0GNU\0\237\333t\347\262\27\320l\223\27*\202C\370T\177"..., 68, 880) = 68
mmap(NULL, 2037344, PROT_READ, MAP_PRIVATE|MAP_DENYWRITE, 5, 0) = 0x7f58afa2b000
mmap(0x7f58afa4d000, 1540096, PROT_READ|PROT_EXEC, MAP_PRIVATE|MAP_FIXED|MAP_DENYWRITE, 5, 0x22000) = 0x7f58afa4d000
mmap(0x7f58afbc5000, 319488, PROT_READ, MAP_PRIVATE|MAP_FIXED|MAP_DENYWRITE, 5, 0x19a000) = 0x7f58afbc5000
mmap(0x7f58afc13000, 24576, PROT_READ|PROT_WRITE, MAP_PRIVATE|MAP_FIXED|MAP_DENYWRITE, 5, 0x1e7000) = 0x7f58afc13000
mmap(0x7f58afc19000, 13920, PROT_READ|PROT_WRITE, MAP_PRIVATE|MAP_FIXED|MAP_ANONYMOUS, -1, 0) = 0x7f58afc19000
close(5)                                = 0
arch_prctl(ARCH_SET_FS, 0x7f58afc1e580) = 0
mprotect(0x7f58afc13000, 16384, PROT_READ) = 0
mprotect(0x55b57358a000, 4096, PROT_READ) = 0
mprotect(0x7f58afc67000, 4096, PROT_READ) = 0
munmap(0x7f58afc1f000, 106506)          = 0
brk(NULL)                               = 0x55b5742ae000
brk(0x55b5742cf000)                     = 0x55b5742cf000
openat(AT_FDCWD, "/usr/lib/locale/locale-archive", O_RDONLY|O_CLOEXEC) = 5
fstat(5, {st_mode=S_IFREG|0644, st_size=8308144, ...}) = 0
mmap(NULL, 8308144, PROT_READ, MAP_PRIVATE, 5, 0) = 0x7f58af23e000
close(5)                                = 0
fstat(1, {st_mode=S_IFCHR|0620, st_rdev=makedev(0x88, 0x1), ...}) = 0
write(1, "Je m'appelle cliquot.\n", 22Je m'appelle cliquot.
) = 22
close(1)                                = 0
close(2)                                = 0
exit_group(0)                           = ?
+++ exited with 0 +++
\end{lstlisting}

On constate que les mots détectés sont : ["printf", "Je m'appelle \%s.$\backslash$\n",
"cliquot"]


\subsection{Question 3}
\label{sec:org786a8ab}

avoir un | pour le stderr (== 2>)
command 2>\&1 \ldots{} | \ldots{}

L'appel systéme pour ouvrir des fichiers est : openat

On l'appelle 3 fois

\lstset{morekeywords={main, printf, include, week_t, action_t, Pile,
File, Arbre},backgroundcolor=\color[rgb]{0.96,0.95,0.98},keywordstyle=\color[rgb]{0.627,0.126,0.941},commentstyle=\color[rgb]{0.233,0.8,0.233},stringstyle=\color[rgb]{1,0,0},keepspaces=true,deletekeywords={ps,scan},basicstyle=\ttfamily,numbers=left,breaklines=true,frame=lines,tabsize=4,language=sh,label= ,caption= ,captionpos=b}
\begin{lstlisting}
command 2>&1 strace -- printf "Je m'appelle %s.\n" $USER | grep openat
\end{lstlisting}

\lstset{morekeywords={main, printf, include, week_t, action_t, Pile,
File, Arbre},backgroundcolor=\color[rgb]{0.96,0.95,0.98},keywordstyle=\color[rgb]{0.627,0.126,0.941},commentstyle=\color[rgb]{0.233,0.8,0.233},stringstyle=\color[rgb]{1,0,0},keepspaces=true,deletekeywords={ps,scan},basicstyle=\ttfamily,numbers=left,breaklines=true,frame=lines,tabsize=4,language=sh,label= ,caption= ,captionpos=b}
\begin{lstlisting}
  openat(AT_FDCWD, "/etc/ld.so.cache", O_RDONLY|O_CLOEXEC) = 5
openat(AT_FDCWD, "/lib/x86_64-linux-gnu/libc.so.6", O_RDONLY|O_CLOEXEC) = 5
openat(AT_FDCWD, "/usr/lib/locale/locale-archive", O_RDONLY|O_CLOEXEC) = 5

\end{lstlisting}


D'après man openat, le file descriptor est le return de la fonction (-1
errno) Ici 5

Les fichiers ouverts sont :
\begin{itemize}
\item /etc/ld.so.cache
\item /lib/x86\textsubscript{64}-linux-gnu/libc.so.6
\item /usr/lib/locale/locale-archive
\end{itemize}

Le mode d'ouverture est read-only (+ un flag que je comprends pas)


\subsection{Question 4}
\label{sec:orga0034c6}

\lstset{morekeywords={main, printf, include, week_t, action_t, Pile,
File, Arbre},backgroundcolor=\color[rgb]{0.96,0.95,0.98},keywordstyle=\color[rgb]{0.627,0.126,0.941},commentstyle=\color[rgb]{0.233,0.8,0.233},stringstyle=\color[rgb]{1,0,0},keepspaces=true,deletekeywords={ps,scan},basicstyle=\ttfamily,numbers=left,breaklines=true,frame=lines,tabsize=4,language=sh,label= ,caption= ,captionpos=b}
\begin{lstlisting}
command 2>&1 strace -- printf "Je m'appelle %s.\n" $USER | grep write
\end{lstlisting}

Un seul write

\subsection{Question 5}
\label{sec:org060bcf9}

Strace était sur le descripteur de fichier stderr

tee : command pour split ce qu'il reçoit en standard input en standard output

\subsubsection{stderr + stdin}
\label{sec:org6860392}

\lstset{morekeywords={main, printf, include, week_t, action_t, Pile,
File, Arbre},backgroundcolor=\color[rgb]{0.96,0.95,0.98},keywordstyle=\color[rgb]{0.627,0.126,0.941},commentstyle=\color[rgb]{0.233,0.8,0.233},stringstyle=\color[rgb]{1,0,0},keepspaces=true,deletekeywords={ps,scan},basicstyle=\ttfamily,numbers=left,breaklines=true,frame=lines,tabsize=4,language=sh,label= ,caption= ,captionpos=b}
\begin{lstlisting}
echo $(strace -- printf "Je m'appelle %s.\n" $USER) 
\end{lstlisting}

\subsubsection{eliminer sortie standard}
\label{sec:org6cea418}

\lstset{morekeywords={main, printf, include, week_t, action_t, Pile,
File, Arbre},backgroundcolor=\color[rgb]{0.96,0.95,0.98},keywordstyle=\color[rgb]{0.627,0.126,0.941},commentstyle=\color[rgb]{0.233,0.8,0.233},stringstyle=\color[rgb]{1,0,0},keepspaces=true,deletekeywords={ps,scan},basicstyle=\ttfamily,numbers=left,breaklines=true,frame=lines,tabsize=4,language=sh,label= ,caption= ,captionpos=b}
\begin{lstlisting}
echo $(strace -- printf "Je m'appelle %s.\n" $USER >/dev/null)
\end{lstlisting}

\subsubsection{redirection stderr > stdout}
\label{sec:org690b966}

\lstset{morekeywords={main, printf, include, week_t, action_t, Pile,
File, Arbre},backgroundcolor=\color[rgb]{0.96,0.95,0.98},keywordstyle=\color[rgb]{0.627,0.126,0.941},commentstyle=\color[rgb]{0.233,0.8,0.233},stringstyle=\color[rgb]{1,0,0},keepspaces=true,deletekeywords={ps,scan},basicstyle=\ttfamily,numbers=left,breaklines=true,frame=lines,tabsize=4,language=sh,label= ,caption= ,captionpos=b}
\begin{lstlisting}
echo -e "$(strace -- printf "Je m'appelle %s.\n" $USER 2>&1)"
\end{lstlisting}

\subsection{Question 6}
\label{sec:org4af69d0}

\lstset{morekeywords={main, printf, include, week_t, action_t, Pile,
File, Arbre},backgroundcolor=\color[rgb]{0.96,0.95,0.98},keywordstyle=\color[rgb]{0.627,0.126,0.941},commentstyle=\color[rgb]{0.233,0.8,0.233},stringstyle=\color[rgb]{1,0,0},keepspaces=true,deletekeywords={ps,scan},basicstyle=\ttfamily,numbers=left,breaklines=true,frame=lines,tabsize=4,language=sh,label= ,caption= ,captionpos=b}
\begin{lstlisting}
strace -e %file -- printf "Je m'appelle %s.\n" $USER
\end{lstlisting}

On obtient :

\lstset{morekeywords={main, printf, include, week_t, action_t, Pile,
File, Arbre},backgroundcolor=\color[rgb]{0.96,0.95,0.98},keywordstyle=\color[rgb]{0.627,0.126,0.941},commentstyle=\color[rgb]{0.233,0.8,0.233},stringstyle=\color[rgb]{1,0,0},keepspaces=true,deletekeywords={ps,scan},basicstyle=\ttfamily,numbers=left,breaklines=true,frame=lines,tabsize=4,language=sh,label= ,caption= ,captionpos=b}
\begin{lstlisting}
  execve("/usr/bin/printf", ["printf", "Je m'appelle %s.\\n", "cliquot"], 0x7fff347673c8 /* 53 vars */) = 0
access("/etc/ld.so.preload", R_OK)      = -1 ENOENT (Aucun fichier ou dossier de ce type)
openat(AT_FDCWD, "/etc/ld.so.cache", O_RDONLY|O_CLOEXEC) = 5
openat(AT_FDCWD, "/lib/x86_64-linux-gnu/libc.so.6", O_RDONLY|O_CLOEXEC) = 5
openat(AT_FDCWD, "/usr/lib/locale/locale-archive", O_RDONLY|O_CLOEXEC) = 5
Je m'appelle cliquot.
\end{lstlisting}

Avec :
\lstset{morekeywords={main, printf, include, week_t, action_t, Pile,
File, Arbre},backgroundcolor=\color[rgb]{0.96,0.95,0.98},keywordstyle=\color[rgb]{0.627,0.126,0.941},commentstyle=\color[rgb]{0.233,0.8,0.233},stringstyle=\color[rgb]{1,0,0},keepspaces=true,deletekeywords={ps,scan},basicstyle=\ttfamily,numbers=left,breaklines=true,frame=lines,tabsize=4,language=sh,label= ,caption= ,captionpos=b}
\begin{lstlisting}
strace -e %file -- bash -c "cat ./temp 2>dev/null/"
\end{lstlisting}

On obtient en plus stat, et access:

\lstset{morekeywords={main, printf, include, week_t, action_t, Pile,
File, Arbre},backgroundcolor=\color[rgb]{0.96,0.95,0.98},keywordstyle=\color[rgb]{0.627,0.126,0.941},commentstyle=\color[rgb]{0.233,0.8,0.233},stringstyle=\color[rgb]{1,0,0},keepspaces=true,deletekeywords={ps,scan},basicstyle=\ttfamily,numbers=left,breaklines=true,frame=lines,tabsize=4,language=sh,label= ,caption= ,captionpos=b}
\begin{lstlisting}
  execve("/usr/bin/bash", ["bash", "-c", "cat ./temp 2>dev/null/"], 0x7ffe25b5fc48 /* 53 vars */) = 0
access("/etc/ld.so.preload", R_OK)      = -1 ENOENT (Aucun fichier ou dossier de ce type)
openat(AT_FDCWD, "/etc/ld.so.cache", O_RDONLY|O_CLOEXEC) = 5
openat(AT_FDCWD, "/lib/x86_64-linux-gnu/libtinfo.so.6", O_RDONLY|O_CLOEXEC) = 5
openat(AT_FDCWD, "/lib/x86_64-linux-gnu/libdl.so.2", O_RDONLY|O_CLOEXEC) = 5
openat(AT_FDCWD, "/lib/x86_64-linux-gnu/libc.so.6", O_RDONLY|O_CLOEXEC) = 5
openat(AT_FDCWD, "/dev/tty", O_RDWR|O_NONBLOCK) = 5
openat(AT_FDCWD, "/usr/lib/locale/locale-archive", O_RDONLY|O_CLOEXEC) = 5
openat(AT_FDCWD, "/usr/lib/x86_64-linux-gnu/gconv/gconv-modules.cache", O_RDONLY) = 5
stat("/home/cliquot/ZZ1/syExp/tp1", {st_mode=S_IFDIR|0775, st_size=4096, ...}) = 0
stat(".", {st_mode=S_IFDIR|0775, st_size=4096, ...}) = 0
stat("/home", {st_mode=S_IFDIR|0755, st_size=4096, ...}) = 0
stat("/home/cliquot", {st_mode=S_IFDIR|0755, st_size=4096, ...}) = 0
stat("/home/cliquot/ZZ1", {st_mode=S_IFDIR|0775, st_size=4096, ...}) = 0
stat("/home/cliquot/ZZ1/syExp", {st_mode=S_IFDIR|0775, st_size=4096, ...}) = 0
stat("/home/cliquot/ZZ1/syExp/tp1", {st_mode=S_IFDIR|0775, st_size=4096, ...}) = 0
stat("/home/cliquot/scripts", {st_mode=S_IFDIR|0775, st_size=4096, ...}) = 0
stat(".", {st_mode=S_IFDIR|0775, st_size=4096, ...}) = 0
stat("/home/cliquot/scripts/bash", 0x7ffd36655a30) = -1 ENOENT (Aucun fichier ou dossier de ce type)
stat("/home/cliquot/scripts/bash", 0x7ffd36655a30) = -1 ENOENT (Aucun fichier ou dossier de ce type)
stat("/home/cliquot/scripts/bash", 0x7ffd36655a30) = -1 ENOENT (Aucun fichier ou dossier de ce type)
stat("/home/cliquot/scripts/bash", 0x7ffd36655a30) = -1 ENOENT (Aucun fichier ou dossier de ce type)
stat("/home/cliquot/scripts/bash", 0x7ffd36655a30) = -1 ENOENT (Aucun fichier ou dossier de ce type)
stat("/home/cliquot/scripts/bash", 0x7ffd36655a30) = -1 ENOENT (Aucun fichier ou dossier de ce type)
stat("/home/cliquot/scripts/bash", 0x7ffd36655a30) = -1 ENOENT (Aucun fichier ou dossier de ce type)
stat("/home/cliquot/.local/bin/bash", 0x7ffd36655a30) = -1 ENOENT (Aucun fichier ou dossier de ce type)
stat("/usr/local/sbin/bash", 0x7ffd36655a30) = -1 ENOENT (Aucun fichier ou dossier de ce type)
stat("/usr/local/bin/bash", 0x7ffd36655a30) = -1 ENOENT (Aucun fichier ou dossier de ce type)
stat("/usr/sbin/bash", 0x7ffd36655a30)  = -1 ENOENT (Aucun fichier ou dossier de ce type)
stat("/usr/bin/bash", {st_mode=S_IFREG|0755, st_size=1183448, ...}) = 0
stat("/usr/bin/bash", {st_mode=S_IFREG|0755, st_size=1183448, ...}) = 0
access("/usr/bin/bash", X_OK)           = 0
stat("/usr/bin/bash", {st_mode=S_IFREG|0755, st_size=1183448, ...}) = 0
access("/usr/bin/bash", R_OK)           = 0
stat("/usr/bin/bash", {st_mode=S_IFREG|0755, st_size=1183448, ...}) = 0
stat("/usr/bin/bash", {st_mode=S_IFREG|0755, st_size=1183448, ...}) = 0
access("/usr/bin/bash", X_OK)           = 0
stat("/usr/bin/bash", {st_mode=S_IFREG|0755, st_size=1183448, ...}) = 0
access("/usr/bin/bash", R_OK)           = 0
stat(".", {st_mode=S_IFDIR|0775, st_size=4096, ...}) = 0
stat("/home/cliquot/scripts/cat", 0x7ffd36655920) = -1 ENOENT (Aucun fichier ou dossier de ce type)
stat("/home/cliquot/scripts/cat", 0x7ffd36655920) = -1 ENOENT (Aucun fichier ou dossier de ce type)
stat("/home/cliquot/scripts/cat", 0x7ffd36655920) = -1 ENOENT (Aucun fichier ou dossier de ce type)
stat("/home/cliquot/scripts/cat", 0x7ffd36655920) = -1 ENOENT (Aucun fichier ou dossier de ce type)
stat("/home/cliquot/scripts/cat", 0x7ffd36655920) = -1 ENOENT (Aucun fichier ou dossier de ce type)
stat("/home/cliquot/scripts/cat", 0x7ffd36655920) = -1 ENOENT (Aucun fichier ou dossier de ce type)
stat("/home/cliquot/scripts/cat", 0x7ffd36655920) = -1 ENOENT (Aucun fichier ou dossier de ce type)
stat("/home/cliquot/.local/bin/cat", 0x7ffd36655920) = -1 ENOENT (Aucun fichier ou dossier de ce type)
stat("/usr/local/sbin/cat", 0x7ffd36655920) = -1 ENOENT (Aucun fichier ou dossier de ce type)
stat("/usr/local/bin/cat", 0x7ffd36655920) = -1 ENOENT (Aucun fichier ou dossier de ce type)
stat("/usr/sbin/cat", 0x7ffd36655920)   = -1 ENOENT (Aucun fichier ou dossier de ce type)
stat("/usr/bin/cat", {st_mode=S_IFREG|0755, st_size=43416, ...}) = 0
stat("/usr/bin/cat", {st_mode=S_IFREG|0755, st_size=43416, ...}) = 0
access("/usr/bin/cat", X_OK)            = 0
stat("/usr/bin/cat", {st_mode=S_IFREG|0755, st_size=43416, ...}) = 0
access("/usr/bin/cat", R_OK)            = 0
stat("/usr/bin/cat", {st_mode=S_IFREG|0755, st_size=43416, ...}) = 0
stat("/usr/bin/cat", {st_mode=S_IFREG|0755, st_size=43416, ...}) = 0
access("/usr/bin/cat", X_OK)            = 0
stat("/usr/bin/cat", {st_mode=S_IFREG|0755, st_size=43416, ...}) = 0
access("/usr/bin/cat", R_OK)            = 0
bash: dev/null/: Aucun fichier ou dossier de ce type
stat("/home/cliquot/.terminfo", 0x55dc92267550) = -1 ENOENT (Aucun fichier ou dossier de ce type)
stat("/etc/terminfo", {st_mode=S_IFDIR|0755, st_size=4096, ...}) = 0
stat("/lib/terminfo", {st_mode=S_IFDIR|0755, st_size=4096, ...}) = 0
stat("/usr/share/terminfo", {st_mode=S_IFDIR|0755, st_size=4096, ...}) = 0
access("/etc/terminfo/x/xterm-256color", R_OK) = -1 ENOENT (Aucun fichier ou dossier de ce type)
access("/lib/terminfo/x/xterm-256color", R_OK) = 0
openat(AT_FDCWD, "/lib/terminfo/x/xterm-256color", O_RDONLY) = 5
--- SIGCHLD {si_signo=SIGCHLD, si_code=CLD_EXITED, si_pid=38458, si_uid=1000, si_status=1, si_utime=0, si_stime=0} ---
+++ exited with 1 +++

\end{lstlisting}

\section{Exercice n°3}
\label{sec:org9f778e8}

\subsection{Question 1}
\label{sec:org999a06c}

L'option qui va nous permettre de tracer les appels systèmes est : -e \%process

Pour tracer les processus fils va être l'option -f

On va chercher à tracer les commandes fork() et exec()

Exemple :

\lstset{morekeywords={main, printf, include, week_t, action_t, Pile,
File, Arbre},backgroundcolor=\color[rgb]{0.96,0.95,0.98},keywordstyle=\color[rgb]{0.627,0.126,0.941},commentstyle=\color[rgb]{0.233,0.8,0.233},stringstyle=\color[rgb]{1,0,0},keepspaces=true,deletekeywords={ps,scan},basicstyle=\ttfamily,numbers=left,breaklines=true,frame=lines,tabsize=4,language=sh,label= ,caption= ,captionpos=b}
\begin{lstlisting}
(strace -e %process -f -- emacs &) 2>&1 | grep 'fork\|exec'
\end{lstlisting}

\lstset{morekeywords={main, printf, include, week_t, action_t, Pile,
File, Arbre},backgroundcolor=\color[rgb]{0.96,0.95,0.98},keywordstyle=\color[rgb]{0.627,0.126,0.941},commentstyle=\color[rgb]{0.233,0.8,0.233},stringstyle=\color[rgb]{1,0,0},keepspaces=true,deletekeywords={ps,scan},basicstyle=\ttfamily,numbers=left,breaklines=true,frame=lines,tabsize=4,language=sh,label= ,caption= ,captionpos=b}
\begin{lstlisting}
  xecve("/usr/bin/emacs", ["emacs"], 0x7ffe1095f6d0 /* 53 vars */) = 0
[pid 39504] vfork(strace: Process 39509 attached
[pid 39509] execve("/usr/bin/gpg", ["/usr/bin/gpg", "--with-colons", "--list-config"], 0x7ffc05bd9090 /* 53 vars */ <unfinished ...>
[pid 39504] <... vfork resumed>)        = 39509
[pid 39509] <... execve resumed>)       = 0
[pid 39504] vfork(strace: Process 39512 attached
[pid 39512] execve("/usr/bin/emacsclient.emacs", ["/usr/bin/emacsclient.emacs", "--version"], 0x7ffc05bd7540 /* 53 vars */ <unfinished ...>
[pid 39504] <... vfork resumed>)        = 39512
[pid 39512] <... execve resumed>)       = 0
[pid 39504] vfork(strace: Process 39513 attached
[pid 39513] execve("/usr/bin/git", ["/usr/bin/git", "config", "--get-all", "credential.helper"], 0x7ffc05bd6c20 /* 54 vars */ <unfinished ...>
[pid 39504] <... vfork resumed>)        = 39513
[pid 39513] <... execve resumed>)       = 0
[pid 39504] vfork(strace: Process 39514 attached
[pid 39514] execve("/usr/bin/git", ["/usr/bin/git", "rev-parse", "--is-inside-work-tree"], 0x7ffc05bd7da0 /* 53 vars */ <unfinished ...>
[pid 39504] <... vfork resumed>)        = 39514
[pid 39514] <... execve resumed>)       = 0
[pid 39504] vfork(strace: Process 39515 attached
[pid 39515] execve("/usr/bin/git", ["/usr/bin/git", "--no-pager", "--literal-pathspecs", "-c", "core.preloadindex=true", "-c", "log.showSignature=false", "-c", "color.ui=false", "-c", "color.diff=false", "rev-parse", "--is-inside-work-tree"], 0x7ffc05bd79f0 /* 54 vars */ <unfinished ...>
[pid 39504] <... vfork resumed>)        = 39515
[pid 39515] <... execve resumed>)       = 0
[pid 39504] vfork(strace: Process 39516 attached
[pid 39516] execve("/usr/bin/git", ["/usr/bin/git", "rev-parse", "--is-inside-work-tree"], 0x7ffc05bd74c0 /* 53 vars */ <unfinished ...>
[pid 39504] <... vfork resumed>)        = 39516
[pid 39516] <... execve resumed>)       = 0
[pid 39504] vfork(strace: Process 39517 attached
[pid 39517] execve("/usr/bin/git", ["/usr/bin/git", "--no-pager", "--literal-pathspecs", "-c", "core.preloadindex=true", "-c", "log.showSignature=false", "-c", "color.ui=false", "-c", "color.diff=false", "rev-parse", "--is-inside-work-tree"], 0x7ffc05bd7110 /* 54 vars */ <unfinished ...>
[pid 39504] <... vfork resumed>)        = 39517
[pid 39517] <... execve resumed>)       = 0
[pid 39504] vfork(strace: Process 39518 attached
[pid 39518] execve("/usr/bin/git", ["/usr/bin/git", "version"], 0x7ffc05bd9f10 /* 54 vars */ <unfinished ...>
[pid 39504] <... vfork resumed>)        = 39518
[pid 39518] <... execve resumed>)       = 0
[pid 39504] vfork(strace: Process 39519 attached
[pid 39519] execve("/usr/bin/hunspell", ["/usr/bin/hunspell", "-vv"], 0x7ffc05bd9900 /* 53 vars */ <unfinished ...>
[pid 39504] <... vfork resumed>)        = 39519
[pid 39519] <... execve resumed>)       = 0
[pid 39504] vfork(strace: Process 39520 attached
[pid 39520] execve("/usr/bin/hunspell", ["/usr/bin/hunspell", "-vv"], 0x7ffc05bd99e0 /* 53 vars */ <unfinished ...>
[pid 39504] <... vfork resumed>)        = 39520
[pid 39520] <... execve resumed>)       = 0
[pid 39504] vfork(strace: Process 39521 attached
[pid 39521] execve("/usr/bin/hunspell", ["/usr/bin/hunspell", "-D", "-a", "/dev/null"], 0x7ffc05bd9930 /* 53 vars */ <unfinished ...>
[pid 39504] <... vfork resumed>)        = 39521
[pid 39521] <... execve resumed>)       = 0
[pid 39504] vfork(strace: Process 39522 attached
[pid 39522] execve("/usr/bin/hunspell", ["/usr/bin/hunspell", "-vv"], 0x7ffc05bd9400 /* 53 vars */ <unfinished ...>
[pid 39504] <... vfork resumed>)        = 39522
[pid 39522] <... execve resumed>)       = 0
[pid 39504] vfork(strace: Process 39523 attached
[pid 39523] execve("/usr/bin/hunspell", ["/usr/bin/hunspell", "-a", "", "-d", "fr_FR", "-i", "UTF-8"], 0x7ffc05be9a50 /* 53 vars */ <unfinished ...>
[pid 39504] <... vfork resumed>)        = 39523
[pid 39523] <... execve resumed>)       = 0
[pid 39504] vfork(strace: Process 39524 attached
[pid 39524] execve("/usr/bin/git", ["/usr/bin/git", "--no-pager", "--literal-pathspecs", "-c", "core.preloadindex=true", "-c", "log.showSignature=false", "-c", "color.ui=false", "-c", "color.diff=false", "rev-parse", "--show-toplevel"], 0x7ffc05bd82f0 /* 54 vars */ <unfinished ...>
[pid 39504] <... vfork resumed>)        = 39524
[pid 39524] <... execve resumed>)       = 0
[pid 39504] vfork(strace: Process 39525 attached
[pid 39525] execve("/usr/bin/git", ["/usr/bin/git", "--no-pager", "--literal-pathspecs", "-c", "core.preloadindex=true", "-c", "log.showSignature=false", "-c", "color.ui=false", "-c", "color.diff=false", "rev-parse", "--git-dir"], 0x7ffc05bd82f0 /* 54 vars */ <unfinished ...>
[pid 39504] <... vfork resumed>)        = 39525
[pid 39525] <... execve resumed>)       = 0
[pid 39504] vfork(strace: Process 39526 attached
[pid 39526] execve("/usr/bin/git", ["/usr/bin/git", "rev-parse", "--is-inside-work-tree"], 0x7ffc05bd8790 /* 53 vars */ <unfinished ...>
[pid 39504] <... vfork resumed>)        = 39526
[pid 39526] <... execve resumed>)       = 0
[pid 39504] vfork(strace: Process 39527 attached
[pid 39527] execve("/usr/bin/git", ["/usr/bin/git", "--no-pager", "--literal-pathspecs", "-c", "core.preloadindex=true", "-c", "log.showSignature=false", "-c", "color.ui=false", "-c", "color.diff=false", "rev-parse", "--is-inside-work-tree"], 0x7ffc05bd83e0 /* 54 vars */ <unfinished ...>
[pid 39504] <... vfork resumed>)        = 39527
[pid 39527] <... execve resumed>)       = 0
\end{lstlisting}

\section{Exercice 4}
\label{sec:org3b4a438}

\subsubsection{redirection fichier}
\label{sec:orgea26ce4}

\begin{itemize}
\item com > fic redirige la sortie standard de com dans le fichier fic,
\item com 2> fic redirige la sortie des erreurs de com dans le fichier fic,
\item com 2>\&1 redirige la sortie des erreurs de com vers la sortie standard de com,
\item com < fic redirige l'entrée standard de com dans le fichier fic,
\item com1 | com2 redirige la sortie standard de la commande com1 vers l'entrée standard de com2.
\item com1 |\& com2 branche ("connecte" selon le manuel bash) la sortie standard et la sortie d'erreur de com1 sur l'entrée de com2

\item Les simples quotes délimitent une chaîne de caractères. Même si cette chaîne
\item contient des commandes ou des variables shell, celles-ci ne seront pas
\item interprétées.
\item Les doubles quotes délimitent une chaîne de caractères, mais les noms de
\item variable sont interprétés par le shell.
\item Bash considère que les Back-quotes délimitent une commande à exécuter. Les
\item noms de variable et les commandes sont donc interprétés.
\end{itemize}

\subsubsection{comparateur}
\label{sec:orgc63bfa5}

\begin{itemize}
\item n1 -eq n2, vrai si n1 et n2 sont égaux (equal) ;
\item n1 -ne n2, vrai si n1 et n2 sont différents (non equal);
\item n1 -lt n2, vrai si n1 est strictement inférieur à n2 (lower than);
\item n1 -le n2, vrai si n1 est inférieur ou égal à n2 (lower or equal);
\item n1 -gt n2, vrai si n1 est strictement supérieur à n2 (greater than) ;
\item n1 -ge n2, vrai si n1 est supérieur ou égal à n2 (greater or equal).

\item ! e, vrai si e est fausse ;
\item e1 -a e2, vrai si e1 et e2 sont vraies ;
\item e1 -o e2, vrai si e1 ou e2 est vraie.
\item 0 est un succès, le reste est un échec.
\end{itemize}

[ expression ] == test expression

\subsubsection{Boucle et structure conditionnelles}
\label{sec:orgbb5a0f1}
\lstset{morekeywords={main, printf, include, week_t, action_t, Pile,
File, Arbre},backgroundcolor=\color[rgb]{0.96,0.95,0.98},keywordstyle=\color[rgb]{0.627,0.126,0.941},commentstyle=\color[rgb]{0.233,0.8,0.233},stringstyle=\color[rgb]{1,0,0},keepspaces=true,deletekeywords={ps,scan},basicstyle=\ttfamily,numbers=left,breaklines=true,frame=lines,tabsize=4,language=sh,label= ,caption= ,captionpos=b}
\begin{lstlisting}
if condition1
    then instruction(s)
elif condition2
    then instruction(s)
elif condition3
    ...
fi
\end{lstlisting}

\lstset{morekeywords={main, printf, include, week_t, action_t, Pile,
File, Arbre},backgroundcolor=\color[rgb]{0.96,0.95,0.98},keywordstyle=\color[rgb]{0.627,0.126,0.941},commentstyle=\color[rgb]{0.233,0.8,0.233},stringstyle=\color[rgb]{1,0,0},keepspaces=true,deletekeywords={ps,scan},basicstyle=\ttfamily,numbers=left,breaklines=true,frame=lines,tabsize=4,language=sh,label= ,caption= ,captionpos=b}
\begin{lstlisting}
  case valeur_testee in
      valeur1) instruction(s);;
      valeur2) instruction(s);;
      valeur3) instruction(s);;
      ) instruction_else(s);;
esac
\end{lstlisting}

\lstset{morekeywords={main, printf, include, week_t, action_t, Pile,
File, Arbre},backgroundcolor=\color[rgb]{0.96,0.95,0.98},keywordstyle=\color[rgb]{0.627,0.126,0.941},commentstyle=\color[rgb]{0.233,0.8,0.233},stringstyle=\color[rgb]{1,0,0},keepspaces=true,deletekeywords={ps,scan},basicstyle=\ttfamily,numbers=left,breaklines=true,frame=lines,tabsize=4,language=sh,label= ,caption= ,captionpos=b}
\begin{lstlisting}
for variable in liste_valeurs
  do instruction(s)
done
\end{lstlisting}

\lstset{morekeywords={main, printf, include, week_t, action_t, Pile,
File, Arbre},backgroundcolor=\color[rgb]{0.96,0.95,0.98},keywordstyle=\color[rgb]{0.627,0.126,0.941},commentstyle=\color[rgb]{0.233,0.8,0.233},stringstyle=\color[rgb]{1,0,0},keepspaces=true,deletekeywords={ps,scan},basicstyle=\ttfamily,numbers=left,breaklines=true,frame=lines,tabsize=4,language=sh,label= ,caption= ,captionpos=b}
\begin{lstlisting}
for i in "$@"
do
    echo "$i"
done
\end{lstlisting}

\lstset{morekeywords={main, printf, include, week_t, action_t, Pile,
File, Arbre},backgroundcolor=\color[rgb]{0.96,0.95,0.98},keywordstyle=\color[rgb]{0.627,0.126,0.941},commentstyle=\color[rgb]{0.233,0.8,0.233},stringstyle=\color[rgb]{1,0,0},keepspaces=true,deletekeywords={ps,scan},basicstyle=\ttfamily,numbers=left,breaklines=true,frame=lines,tabsize=4,language=sh,label= ,caption= ,captionpos=b}
\begin{lstlisting}
while condition
do
    instruction(s)
done
\end{lstlisting}

\lstset{morekeywords={main, printf, include, week_t, action_t, Pile,
File, Arbre},backgroundcolor=\color[rgb]{0.96,0.95,0.98},keywordstyle=\color[rgb]{0.627,0.126,0.941},commentstyle=\color[rgb]{0.233,0.8,0.233},stringstyle=\color[rgb]{1,0,0},keepspaces=true,deletekeywords={ps,scan},basicstyle=\ttfamily,numbers=left,breaklines=true,frame=lines,tabsize=4,language=sh,label= ,caption= ,captionpos=b}
\begin{lstlisting}
a=1
a=$(($a + 1))
echo $a

a=1
let "a=$a + 1"
echo $a

a=1
a=$(echo "$a+1" |bc )
echo $a

declare -i name=...
\end{lstlisting}

\section{Exercice 5 (Coprocesses)}
\label{sec:org6fc69f5}

The standard output of command is connected via a pipe to a file descriptor
the executing shell, and that file descriptor is assigned to NAME[0].  The  standard  input
of  command  is  connected via a pipe to a file descriptor in the executing shell, and that
file descriptor is assigned to NAME[1]. (man Bash coprocesses)



\begin{itemize}
\item < lecture
\item <> lect et ecriture
\item >> append
\item > écriture
\end{itemize}

>name = nom de fichier
>\&num = descripteur de fichier
num>- ferme le descripteur

ARRAY :
+\(TAB = première case
  +\)\{TAB[n]\} = contenu de la case n° n.
+\$\{TAB[@]\} = tout le contenu.

\lstset{morekeywords={main, printf, include, week_t, action_t, Pile,
File, Arbre},backgroundcolor=\color[rgb]{0.96,0.95,0.98},keywordstyle=\color[rgb]{0.627,0.126,0.941},commentstyle=\color[rgb]{0.233,0.8,0.233},stringstyle=\color[rgb]{1,0,0},keepspaces=true,deletekeywords={ps,scan},basicstyle=\ttfamily,numbers=left,breaklines=true,frame=lines,tabsize=4,language=sh,label= ,caption= ,captionpos=b}
\begin{lstlisting}
echo "bonjour" >&${COPROC[1]}
echo <&${COPROC[0]}
\end{lstlisting}

Pour créer le signal, il n'exite pas de symbole EOF, mais on peut fermer le
descripteur de fichier pour simuler le EOF.

\lstset{morekeywords={main, printf, include, week_t, action_t, Pile,
File, Arbre},backgroundcolor=\color[rgb]{0.96,0.95,0.98},keywordstyle=\color[rgb]{0.627,0.126,0.941},commentstyle=\color[rgb]{0.233,0.8,0.233},stringstyle=\color[rgb]{1,0,0},keepspaces=true,deletekeywords={ps,scan},basicstyle=\ttfamily,numbers=left,breaklines=true,frame=lines,tabsize=4,language=sh,label= ,caption= ,captionpos=b}
\begin{lstlisting}
eval exec ${COPROC[1]}">&-"
\end{lstlisting}


On a:
exec qui va permettre de considérer la redirection comme une commande.
eval qui va remplacer les variables AVANT d'exécuter.
le >\&- entre guillemet pour éviter que shell le remplace
\end{document}